\newglossaryentry{token}{
	name={\emph{token}},
	description={Une unité de texte traitée comme une séquence distincte par le modèle. Il peut correspondre à un mot, une partie de mot ou un symbole}
}


\newglossaryentry{quantization}{
	name={quantisation},
	description={La quantisation consiste à réduire le nombre de bits utilisés pour représenter les poids (para-
		mètres ajustables qui déterminent la force des connexions entre les neurones) et les activations (valeurs
		intermédiaires produites par ces neurones à chaque étape du traitement) d’un modèle de langage. Par
		exemple, au lieu d’utiliser des nombres à virgule de 32 bits (float32), on peut utiliser des entiers de 8 bits
		(int8). Cela permet de diminuer la mémoire nécessaire pour stocker le modèle et d’accélérer les calculs
		nécessaires pour faire des inférences}
}

\newglossaryentry{LLM}{
	name={\emph{LLM (Large language models)}},
	description={Modèle d'intelligence artificielle entraîné sur de vastes quantités de texte pour comprendre, générer et manipuler du langage naturel}
}

\newglossaryentry{fine-tuning}{
	name={\emph{fine-tuning}},
	description={Processus d'ajustement d'un modèle d'intelligence artificielle pré-entraîné sur un ensemble de données spécifique afin d'améliorer ses performances pour une tâche particulière}
}

\newglossaryentry{chatbot}{
	name={\emph{chatbot}},
	description={Programme informatique conçu pour simuler une conversation avec des utilisateurs humains, généralement via une interface de messagerie ou vocale, en utilisant des règles préprogrammées ou des modèles d'intelligence artificielle}
}

\newglossaryentry{TAL}{
	name={TAL (Traitement Automatique du Langage)},
	description={Le traitement automatique du langage ou \emph{natural language processing} en anglais, est un domaine de l'intelligence artificielle qui se concentre sur l'interaction entre les ordinateurs et le langage humain, en permettant l'analyse, la compréhension et la génération de texte ou de parole}
}
\newglossaryentry{NLP}{
	name={\emph{NLP (Natural langage processing)}},
	description={Voir TAL (Traitement Automatique du Langage)}
}


\newglossaryentry{stop-words}{
	name={\emph{stop words}},
	description={Mots courants dans une langue (comme "le", "et", "de" en français) qui sont souvent filtrés ou ignorés lors du traitement de texte en \emph{machine learning} car ils apportent peu de valeur informationnelle pour l'entraînement ou l'analyse}
}

\newglossaryentry{RAG}{
	name={RAG (Retrieval Augmented Generation)},
	description={Technique permettant de baser les réponses d'un LLM sur des informations contenues au sein d’une base de connaissance externe. Des informations pertinentes pour générer une réponse à un prompt sont récupérées dans la base et ajoutées à la fenêtre du LLM, qui s'appuie dessus pour générer sa réponse. Cette technique permet des réponses plus précies et sourcées}
}

\newglossaryentry{vectorisation}{
	name={vectorisation},
	description={Processus de transformation de données (textuelles ou non) en vecteurs numériques afin qu'elles puissent être traitées par des algorithmes de \emph{machine learning}}
}

\newglossaryentry{prompt}{
	name={prompt},
	description={Instruction ou texte d'entrée donné à un modèle d'intelligence artificielle génératif pour générer une réponse ou accomplir une tâche spécifique}
}

\newglossaryentry{précision}{
	name={précision},
	description={Proportion des prédictions positives correctes parmi toutes les prédictions positives faites par un modèle, mesurant ainsi sa capacité à éviter les faux positifs}
}

\newglossaryentry{rappel}{
	name={rappel},
	description={Proportion des prédictions positives correctes parmi tous les cas réellement positifs dans les données, indiquant ainsi la capacité du modèle à identifier les vrais positifs}
}

\newglossaryentry{F1-score}{
	name={F1 score},
	description={Le F1-score est une mesure de performance qui combine précision et rappel. 
		Il s'agit d'une moyenne entre les deux se situant entre 0 et 1, 1 étant la meilleure
		performance possible}
}

\newglossaryentry{benchmark}{
	name={\emph{benchmark}},
	description={Une référence ou un ensemble de critères utilisés pour évaluer la performance, la qualité ou l'efficacité d'un système, d'un modèle, d'un produit, en le comparant à un standard ou à d'autres systèmes similaires}
}

\newglossaryentry{fenêtre}{
	name={fenêtre},
	description={Quantité de texte (souvent mesurée en tokens ou caractères) qu'un modèle de langage peut traiter ou dont il peut se souvenir à un moment donné, influençant la cohérence et la pertinence de ses réponses}
}

\newglossaryentry{hallucination}{
	name={hallucination},
	description={En machine learning, une hallucination se produit lorsqu'un modèle, comme un LLM, génère des informations incorrectes, inventées ou non fondées, souvent présentées comme étant factuelles}
}

\newglossaryentry{biais}{
	name={biais algorithmiques},
	description={Préjugés ou partialités intégrées dans un modèle d'intelligence artificielle, souvent résultant de données d'entraînement non représentatives, qui peuvent conduire à des décisions ou des prédictions inéquitables}
}


\newglossaryentry{framework}{
	name={framework},
	description={Ensemble structuré d'outils, de bibliothèques, et de conventions utilisé pour développer des applications logicielles, fournissant un cadre pour accélérer et standardiser le développement}
}

\newglossaryentry{responsive}{
	name={responsive},
	description={Terme utilisé pour décrire des interfaces web ou des applications capables de s'adapter automatiquement à différentes tailles d'écran et dispositifs (ordinateurs, tablettes, smartphones) pour offrir une meilleure expérience utilisateur}
}


\newglossaryentry{apprentissage}{
	name={apprentissage machine ou \emph{machine learning}},
	description={Branche de l'intelligence artificielle qui permet d'entraîner un système à partir de données, d'améliorer ses performances dans le but de lui faire faire des prédictions ou prendre des décisions sans être exactement programmé pour chaque tâche}
}

\newglossaryentry{IA}{
	name={intelligence artificielle (IA)},
	description={Systèmes informatiques qui seraient capables de simuler des
		capacités cognitives humaines, comme l'apprentissage et la prise de
		décision, afin de traiter et analyser des données à grande échelle}
}

\newglossaryentry{générative}{
	name={IA générative},
	description={Branche de l’intelligence artificielle dont les modèles créent de nouvelles données, telles que du texte, des images ou de la musique}
}

\newglossaryentry{OCR}{
	name={\emph{OCR (optical character recognition)}},
	description={Technologie qui lit et convertit des images de texte dactylographié ou imprimé en texte numérique}
}

\newglossaryentry{HTR}{
	name={\emph{HTR (Handwritten Text Recognition)}},
	description={Technologie similaire à l'\emph{OCR}, mais spécifiquement conçue pour reconnaître et convertir le texte manuscrit en format numérique}
}

\newglossaryentry{REGEX}{
	name={expressions régulières (REGEX)},
	description={Ensemble de motifs utilisés pour identifier, rechercher, ou manipuler des chaînes de caractères selon des règles syntaxiques définies}
}

\newglossaryentry{speech}{
	name={\emph{speech to text}},
	description={Technologie qui convertit la parole en texte écrit en temps réel à l’aide de systèmes de reconnaissance vocale}
}

\newglossaryentry{agiles}{
	name={méthodes agiles},
	description={Ensemble de pratiques de gestion de projet et de développement logiciel centrées sur l'adaptabilité, la collaboration, l'itération et la réponse rapide aux changements de besoins}
}

\newglossaryentry{supervisé}{
	name={apprentissage supervisé},
	description={Type d’intelligence artificielle dont les modèles ou algorithmes apprennent à partir de données étiquetées, où chaque entrée est associée à une sortie considérée correcte. L’algorithme utilise ces exemples pour identifier des motifs et générer des prédictions ou des décisions sur de nouvelles données similaires}
}

\newglossaryentry{non-supervisé}{
	name={apprentissage non supervisé},
	description={Type d'apprentissage où un modèle est entraîné sur des données non étiquetées, c’est-à-dire
		sans indication préalable des réponses ou catégories. L’algorithme doit identifier des structures, des motifs
		ou des regroupements dans les données de manière autonome.}
}

\newglossaryentry{pré-entraîné}{
	name={modèle pré-entraîné},
	description={Modèle d’intelligence artificielle qui a été initialement formé sur un grand ensemble de données
		pour accomplir une tâche générale, avant d’être affiné ou adapté pour une tâche spécifique avec moins de
		données. Ce processus permet de bénéficier de l’apprentissage préalable, accélérant le développement et
		améliorant les performances du modèle sur des applications spécialisées}
}

\newglossaryentry{changement}{
	name={conduite du changement},
	description={Processus d'accompagnement des individus et des organisations dans la transition d'un état actuel vers un état futur désiré, en minimisant les résistances et en maximisant l'adhésion aux nouvelles méthodes, outils ou structures}
}

\newglossaryentry{chaîne}{
	name={chaîne de traitement},
	description={Séquence d'étapes ou d'opérations par lesquelles les données ou les tâches passent afin d'être transformées, analysées ou traitées jusqu'à obtenir un résultat final}
}



\newglossaryentry{clustering}{
	name={classification automatique},
	description={Processus par lequel un algorithme attribue automatiquement des catégories ou des étiquettes à des données en fonction de leurs caractéristiques}
}

\newglossaryentry{NER}{
	name={\emph{NER (Named Entity Recognition)}},
	description={Technique en traitement automatique du langage (TAL) qui identifie et classe automatiquement les entités nommées (comme les personnes, les organisations, les lieux) dans un texte}
}

\newglossaryentry{NEL}{
	name={\emph{NEL (Named Entity Linking)}},
	description={Processus en TAL qui consiste à associer les entités nommées identifiées dans un texte à des entités spécifiques dans une base de données ou un référentiel de connaissances}
}

\newglossaryentry{topic}{
	name={\emph{topic modelling}},
	description={Technique de machine learning utilisée pour découvrir automatiquement les thèmes ("topics") latents présents dans un ensemble de documents}
}

\newglossaryentry{deep}{
	name={deep learning},
	description={Sous-domaine du machine learning utilisant des réseaux de neurones profonds pour modéliser et résoudre des problèmes complexes à partir de grandes quantités de données}
}


\newglossaryentry{cloud}{
	name={\emph{cloud computing}},
	description={Fourniture de services informatiques (serveurs, stockage, bases de données, etc.) à distance via une connexion
		réseau, permettant un accès flexible et évolutif aux ressources numériques}
}

\newglossaryentry{litteracie}{
	name={littératie numérique},
	description={Capacité à utiliser de manière critique, sécurisée et efficace les technologies numériques pour accéder à l'information, communiquer, créer du contenu et résoudre des problèmes}
}


\newglossaryentry{multimodal}{
	name={multimodal},
	description={En intelligence artificielle, le terme désigne la capacité d'un modèle à intégrer et traiter plusieurs types de données ou de médias (texte, image, audio, etc.) pour effectuer des tâches complexes ou fournir des réponses plus précises}
}

\newglossaryentry{découvrabilité}{
	name={découvrabilité},
	description={Facilité avec laquelle les utilisateurs peuvent trouver et accéder à des informations, des produits
		ou des services, souvent en lien avec la conception de l’interface utilisateur ou l’optimisation des moteurs
		de recherche}
}


\newglossaryentry{inference}{
	name={inférence},
	description={Processus par lequel le modèle génère des réponses ou des prédictions en s'appuyant sur son entraînement préalable pour traiter et interpréter des données qui lui ont été transmises}
}


\newglossaryentry{embeddings}{
	name={\emph{embeddings}},
	description={Représentations vectorielles de mots ou de phrases dans le but de capter leurs significations et relations
		contextuelles dans un espace de grande dimension}
}
