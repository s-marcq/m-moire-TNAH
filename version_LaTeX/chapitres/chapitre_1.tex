\chapter{Un contexte européen et luxembourgeois favorable au développement de projets IA dans le secteur public}

\subsection{Ambitions et bénéfices pour les administrations publiques}


L'innovation dans le domaine de l'intelligence artificielle est poussée par l'Union européenne et l'État luxembourgeois. Les
ambitions sont élevées et, en cas de succès, les bénéfices peuvent être
nombreux pour le pays et ses administrations.

Les publicités pour les téléphones portables, ordinateurs et voitures
intégrant de l\textquotesingle IA se multiplient.
L\textquotesingle Intelligence artificielle envahit notre quotidien et
est devenue un véritable argument marketing. Elle est entourée
d\textquotesingle une mythologie la présentant comme la solution à de
nombreux problèmes. L'\gls{générative} a été introduite au grand public par
l'arrivée ChatGPT en novembre 2022. C'est à ce moment que ce dernier a pour la première fois réellement pu s'approprier une technologie basée sur de
l'\gls{apprentissage}. Les possibilités semblent infinies lorsqu'on
commence à utiliser les IA génératives les plus puissantes du marché, mais la plus grande partie des utilisateurs n'a pas d'idée
précise de ce qu'il y a derrière. Le fonctionnement des IA génératives
est obscur pour le Grand public et pour beaucoup d'administrations, dont
le personnel n'est souvent pas encore formé. Les possibilités semblent
donc infinies mais leurs usages sont difficiles à définir concrètement.
Un article de blog de l'anthropologue Madeleine Clare Elish aborde
l'idée de magie liée à l'intelligence artificielle\footcite{elish_dont_2018}.
Ce champs lexical de la magie souvent utilisé pour décrire ces
technologies est révélateur de leur opacité, la magie étant
quelque chose qui produit un résultat mais qui ne s'explique
pas. L'intelligence artificielle est associée à un certain nombre de
fantasmes, à une forme de mythologie\footcite{duca_artificial_2023}. Cette dernière est
héritée de la science fiction, dans laquelle l'imaginaire des machines
dont les capacités égaleraient celles des humains et prendraient le
contrôle a été vivement exploité. Le terme-même d'Intelligence
artificielle est un symptôme de ces fantasmes en évoquant
l\textquotesingle idée d'une intelligence comparable à celle des humains
pour les machines, nourrissant les mythes et peurs véhiculés par la
science-fiction. Le terme d'\gls{apprentissage} est souvent plus
adapté mais reste peu utilisé car moins impactant. Cette mythologie qui
entoure l'intelligence artificielle la rend ainsi source de grandes
ambitions pour les acteurs publics ou privés. Elle
est fréquemment exploitée dans les discours marketing qui contribuent à
entretenir des attentes irréalistes et des perceptions erronées de ce
que l\textquotesingle IA peut accomplir et de son
fonctionnement\footcite{duca_artificial_2023}. Ces ambitions et cette
obscurité se sont ressenties au lancement du projet \emph{InventAIre},
dont les ambitions sont assez élevées et surtout peu cadrées au départ
parce que le projet a mûri au sein d'une équipe et d'une administration
qui n'avaient pas encore une grande maîtrise des technologies de
\emph{machine learning}. La première partie de notre stage a ainsi été
consacrée à un travail de définition précis d'un périmètre. Créer un outil
capable de remplir l'intégralité des colonnes de l'inventaire des
Archives nationales était impossible en quatre mois. Les ambitions
élevées et le fait que l'intelligence artificielle soit un sujet
d'actualité peuvent être néanmoins bénéfiques en facilitant le
financement de projets IA, dans un secteur public où ils sont souvent complexes
à obtenir. Plusieurs bénéfices de l'intelligence artificielle dans les
administrations publiques ont été identifiés dans un article récemment paru
dans la revue \emph{Gestion et management public} à partir de la
littérature scientifique sur cette dernière\footcite{bertolucci_lintelligence_2024}.
Nous pouvons les regrouper en quatre catégories~: efficacité et gestion
des ressources, sécurité, transparence, qualité de service. Ce sont
surtout les gains économiques, obtenus grâce à l'automatisation des
tâches redondantes, qui sont les plus intéressants dans un service
public cherchant à réaliser des économies. Ambitions et potentiels bénéfices sont
ainsi élevés au sein des administrations.
\newline

À l'échelle des états, une «~course à l'IA~» est en marche depuis
quelques années et les pousse à investir dans ces technologies. Un
article de Charles Thibout, chercheur en sciences politiques, publié en
2018 fait remonter le début de cette course au tournant des années
2010\footcite{thibout_intelligence_2019}.
Les états de l'Union européenne ne veulent pas se sentir dépassés par
ces nouvelles technologies. À l'échelle de l'Union, une stratégie en
matière d'intelligence artificielle a été rédigée en 2018 par la
Commission européenne\footcite{noauthor_communication_2018}. Les bénéfices mis en avant pour les pays de
l'UE sont d'ordre éthique~: les outils produits dans ces pays ont
beaucoup plus de chances d'être conformes aux règles européennes. Les enjeux
sont toutefois principalement économiques~: d'après plusieurs économistes
l'intelligence artificielle pourrait permettre de stimuler la
croissance\footcite{aghion_intelligence_2019}. Ces systèmes
vecteurs de croissance ont donc une grande valeur. Un pays pourra générer
beaucoup de bénéfices en exportant ces innovations. Les états, et
plus largement le secteur public, souhaitent par ailleurs anticiper les
évolutions numériques pour éviter d\textquotesingle être dépassés. S\textquotesingle engager activement dans le développement et
l\textquotesingle innovation permet d'éviter de subir une transition numérique trop
précipitée. Il est dans leur intérêt de ne pas dépendre du secteur
privé. Dans le cas de l'Union européenne ne pouvant rivaliser avec les
grandes puissances telles que la Chine et les États-Unis, développer des
compétences, infrastructures et outils techniques contribue à limiter la
dépendance envers ces dernières. De plus, pour les mêmes raisons, l'accent est mis sur un développement éthique et responsable de l'IA,
des valeurs qui s'intègrent bien aux ambitions des institutions
publiques\footcite{smuha_race_2021}.

Le Luxembourg n'échappe pas à cette course vectrice d'investissements.
Une stratégie nationale pour l'IA a été publiée en mai 2019, un peu plus
d'un an après la publication de la stratégie française\footnote{Le
	rapport de Cédric Villani intitulé «~Donner un sens à l'intelligence
	artificielle. Pour une stratégie nationale et européenne~» est publié
	en mars 2018.}. Une partie de la stratégie du Grand-Duché est
consacrée à «~L'IA au service du secteur public~». Il y est expliqué que
les systèmes IA peuvent améliorer la qualité des services. Des actions
sont prévues pour pousser le développement de tels systèmes~: évaluation
des projets potentiels, échanges avec des autres états membres de
l'Union européenne, promotion de la recherche et de l'innovation,
développement de solutions pour l'administration et développement de
bases de données publiques\footcite{noauthor_intelligence_2019}. Des investissements ont été
réalisés. Un superordinateur d'une valeur de 30,4 millions d'euros a par
exemple été inauguré en 2021 dans le but de pouvoir traiter des grands
volumes de données et entraîner des modèles de \emph{machine learning}\footcite{noauthor_meluxina_2023}.
Le secteur public luxembourgeois présente certains avantages qui le
rendent propice au développement de l'IA. Il apparaît comme un meilleur
garant de l'éthique des algorithmes que le secteur privé. L'IA est
définie comme «~une technologie puissante, entièrement sous notre
contrôle, et débordante de possibilités~»\footcite{noauthor_intelligence_2019} dans
l'introduction de la stratégie nationale rédigée par Xavier Bettel,
alors Premier ministre et ministre de la Digitalisation. L'expression
«~sous notre contrôle~» souligne l\textquotesingle importance accordée à
une gouvernance responsable et à une maîtrise des impacts de
l\textquotesingle IA. 

Ainsi, le contexte actuel d'émulation, la mythologie qui entoure
l'intelligence artificielle et ses bénéfices pressentis encouragent le
secteur public à investir dans ce type de technologies. Il subsiste
malgré tout des inquiétudes qui poussent les états de l'Union européenne
vers des questionnements d'ordre éthique et vers la mise en place de
cadres régulateurs.


\subsection{La mise en place d'un cadre propice au développement de l'IA dans les institutions publiques}

Comme évoqué précédemment, l'Union européenne a pris le virage de la
responsabilité dans la course à l'intelligence artificielle. La question
des risques était présente dès les débuts de la théorisation d'une
potentielle intelligence des machines par des chercheurs tels qu'Alan Turing et Irving John Good. Leurs craintes étaient centrées sur un potentiel dépassement de
l'intelligence humaine par la machine\footcite{beard_9_2023}. L'imprévisibilité de
la machine et le fait qu'elle ne soit pas dotée de sentiments
inquiétait. Le développement des \gls{générative}s a soulevé des
préoccupations concernant la désinformation, la manipulation de
l\textquotesingle opinion publique, le respect du droit d'auteur, du
RGPD (Règlement général de protection des données) et les biais
algorithmiques. Face à ces risques, un cadre est à mettre en place. 

La question de la réglementation ne date pas d'hier. Elle est abordée dans
plusieurs stratégies nationales, dont celle du Luxembourg.

\begin{quote}
	Compte tenu de l'importance stratégique et de la grande complexité de ce
	sujet, le Luxembourg tient à investir dans un cadre amélioré propice à
	l'IA. Cet objectif implique d'envisager une nouvelle réglementation,
	garantissant un marché des données fonctionnel, par exemple afin
	d'éliminer les obstacles au développement d'une IA fiable\footcite{noauthor_intelligence_2019}
\end{quote}

Toutefois, il a fallu attendre que les systèmes se démocratisent,
notamment via les IA génératives, ayant suscité de vives d'inquiétudes,
pour que la question soit étudiée plus en profondeur. L'Union européenne
a commencé le travail avec l'\emph{AI Act}, publié au Journal officiel
le 12 juillet 2024. Ce règlement de l'UE concernant l'intelligence
artificielle est une première dans monde. Il part d'une approche basée
sur les risques, classés en cinq catégories~: inacceptable, haut risque,
risque spécifique, risque associé aux IA d'usage général et risque
systémique associé aux IA d'usage général. Le règlement interdit les
systèmes à risque inacceptable. Une liste de ces derniers est fournie
dans l'article 5. Ils incluent~:

	\begin{quote}
		La manipulation cognitivo-comportementale de personnes ou de groupes
		vulnérables spécifiques : par exemple, des jouets activés par la voix
		qui encouragent les comportements dangereux chez les enfants
	\end{quote}

	\begin{quote}
		Un score social : classer les personnes en fonction de leur
		comportement, de leur statut socio-économique, de leurs
		caractéristiques personnelles
	\end{quote}

	\begin{quote}
		Une catégorisation et une identification biométriques des personnes
	\end{quote}

	\begin{quote}
		Des systèmes d\textquotesingle identification biométrique en temps
		réel et à distance, tels que la reconnaissance faciale\footcite{loi_ia}
	\end{quote}


Les systèmes à hauts risques doivent être évalués régulièrement et les
systèmes risques limités ont des obligations de transparence à respecter
et doivent respecter le droit d'auteur. Cette approche par risque a
l'avantage d'être assez vague pour traiter des grandes menaces dont on
ne peut pas encore forcément prédire la forme, mais peut néanmoins
sembler assez floue. Le monde de l'IA est voué à évoluer. D'autres
législations sont à prévoir même s'il s'agit d'une base importante d'établissement d'un cadre légal pour une IA plus responsable.
\newline

Des cadres se construisent également à l\textquotesingle échelle
institutionnelle afin de guider les futurs projets et les futurs usages
du personnel des administrations et de leur public. La
question de la confiance des utilisateurs envers l'IA est récurrente
dans les guides et documents d'études émanant des états. Une étude du
Conseil d'État français datant de 2022 associait dans son titre la
confiance à la performance~: «~Intelligence artificielle et action
publique : construire la confiance, servir la performance\footcite{detat_intelligence_2022}~». 
En effet, la confiance permet d'aller jusqu'au bout des projets
et assure une utilisation des outils IA développés. Dans le rapport
d'une «~Consultation publique relative aux opportunités et aux défis de
l'Intelligence Artificielle~» datant de 2021 menée par le \emph{LISER
(Luxembourg Institute of Socio-economic Research)}, 58~\% des personnes
interrogées avaient une confiance moyenne dans une IA mise en œuvre dans
le secteur public contre 41~\% dans le privé\footcite{poussing_resultats_2021}.
Le secteur public luxembourgeois a pour avantage d'être vu comme un
secteur plus cadré, qui priorise davantage l'éthique que le secteur
privé. Cela facilite la confiance de son public. Toutefois, les usagers
ne sont pas les seules personnes concernées par l'IA. Le personnel des
administrations est le premier acteur humain impliqué dans les
processus. La mise en place d'un cadre devrait permettre une meilleure
confiance de sa part en les outils IA qu'ils seront amenés à utiliser.
Cette idée est évoquée à propos de l'usage de l'IA pour le traitement
des archives numériques dans un article récent intitulé «~Applying AI to
digital archives: trust, collaboration and shared professional
ethics~»\footcite{jaillant_applying_2023}. Les
auteurs y expliquent que l'IA peut être un outil performant pour les
archivistes, mais pour exploiter son potentiel, les professionnels
doivent être d'accord sur ce qui est éthique et ce qui ne l'est pas.
Ils proposent une collaboration des différents acteurs~: producteurs
d'archives, professionnels des archives et chercheurs, pour développer
des codes de conduite.
Pour guider l'usage de l'IA et assurer cette performance dans le secteur
public, des chartes ont été rédigées ou sont en cours de
rédaction. Cela a été le cas à la Chambre des Députés, où une
charte IA a été publiée fin juillet 2024. Elle expose «~10
lignes directrices que la Chambre des Députés suivra pour ses futurs
projets en lien avec l'intelligence artificielle, notamment en matière
de transparence, d'éthique et de responsabilité\footcite{noauthor_chambre_nodate}~». Elle a été
rédigée par une équipe composée de personnel de différents services,
dont le service informatique, la Cellule archives, la Cellule scientifique, ou encore le Service du compte-rendu.
Les acteurs sont donc divers, regroupant des personnes des métiers
traditionnels de l'administration parlementaire, du monde de
l'informatique, de la recherche et les archivistes. Cette collaboration
lui octroie une plus grande légitimité et visibilité
au sein de l'administration. Elle est courte et facilement
compréhensible. La charte est un premier cadre qui permettra le
développement croissant de projets IA et facilitera leur mise en
production. Il s'agit en quelque sorte d'une étape de fondation de la
politique de \gls{changement} sur le sujet. La charte de la
Chambre a donné lieu à plusieurs articles dans les médias
luxembourgeois. Elle peut aussi être une forme de vitrine pour les
administrations publiques, les présentant comme modernes et à la pointe
de l\textquotesingle innovation.

Ces différents cadres devraient permettre la mise en place de projets IA
de manière plus sûre et avec une confiance plus accrue du personnel et
des utilisateurs des outils développés. Grâce à ces structures, les
projets IA peuvent désormais être élaborés avec une approche plus
rigoureuse et transparente. Il s'agit d'un grand avantage au sein des
parlements, où s'écrivent et se votent normalement l'établissement de
ces cadres, et où ces initiatives d'intelligence artificielle commencent
à se déployer.


\subsection{Le cas des parlements~: vers les premières mises en production d'outils basés sur l'IA}

Les projets pilotes IA se multiplient dans les parlements. Un grand
nombre a été présenté lors d'un séminaire intitulé «~Use of artificial
intelligence for parliamentary research and documentation~» organisé par
l'ECPRD (\emph{European Center for Parliamentary Research and
	Documentation}) à Rome en mai 2024. L'intelligence artificielle est un
grand sujet de discussion dans les administrations parlementaires. Dans
le \emph{Bulletin de l'innovation} de l'Union interparlementaire
(l\textquotesingle organisation internationale des parlements) d'octobre
2023, trois personnes interrogées, membres des parlements européens,
brésiliens et grecs, insistaient sur l\textquotesingle importance du
réseau inter-parlementaire pour soutenir les initiatives IA, coopérer et
partager des expériences\footcite{noauthor_innovation_nodate}. L'Union inter-parlementaire, l'ECPRD, et les
pratiques d'échanges entre parlements permettent en effet de mutualiser
les connaissances sur des technologies encore récentes dans ces
institutions. 

Différents projets ont été poussés suite à la montée de l'IA générative. Les grands modèles de
langage génératifs sont pré-entraînés et nécessitent ainsi la mise à disposition de moins de
données et moins de connaissances techniques. Parmi ces projets pilotes d'envergure en contexte législatif, nous pouvons citer le projet
\emph{LlaMandement}, mené par la Direction Générale des Finances Publiques en France, qui présente des
similarités avec le projet \emph{InventAIre} de la Chambre des Députés.
Il s'agit de générer automatiquement des résumés neutres d'amendements
législatifs. Pour cela, le grand modèle de langage \emph{Llama 2}
développé par l'entreprise Meta a été
\emph{fine-tuné}\footcite{gesnouin_llamandement_2024}, c'est à dire qu'il a été
ajusté et affiné sur un ensemble spécifique de données pertinentes pour
le traitement des amendements. Cette phase de \gls{fine-tuning} permet
au \gls{pré-entraîné} d'adapter ses réponses et ses capacités de synthèse aux exigences
particulières des résumés législatifs. Cette idée de générer des résumés
fait écho au remplissage des colonnes «~titre~» et «~description~» de
notre inventaire. L'étendue du travail réalisé par l'équipe montre que
l'acte de résumer n'est pas neutre dans un contexte législatif et que si
l'on voulait obtenir les meilleurs descriptions et les meilleurs titres
possibles, il faudrait idéalement être en mesure de \emph{fine-tuner} un
modèle, ce qui était pour nous impossible en quatre mois avec des
données non étiquetées et le matériel dont nous disposions. Les données ayant servi au \emph{fine-tuning} dans
le cadre du projet \emph{LlaMandement} ont été postés sur le web, elles
contiennent un peu plus de 9 000 documents avec leur résumé\footnote{Les données
	du projet LlaMandement postées sur Gitlab sont accessibles via cet url :
	\url{https://gitlab.adullact.net/dgfip/projets-ia/llamandement}}. Les
apprentissages du projet \emph{LlaMandement} ont ainsi pu guider notre
approche au début du projet \emph{InventAIre}, ce qui illustre à quel
point la mutualisation des savoirs entre les institutions
menant des projets IA en contexte législatif est importante.

Dans le domaine des archives, quelques projets ont été mis en production dans des parlements. Ce sont souvent des
projets de reconnaissance automatique de caractères sur des documents : \gls{OCR}, ou \gls{HTR}. Ce sont des technologies
plus anciennes, donc davantage maîtrisées et l'impact est assez faible
en cas d'erreur. En dehors des projets d'\emph{OCR}, c'est le service des
archives du parlement européen qui paraît le plus actif, autour de
l'équipe de Ludovic Delépine. L'IA y est actuellement utilisée pour
classifier automatiquement des documents, générer des résumés et
faciliter la recherche dans les archives. Ils ont mis en
production en avril 2024 \emph{Archibot 3.0}, \gls{chatbot} permettant de
faire une recherche en langage naturel dans un corpus d'un peu plus de
450 000 documents\footcite{kimaid_artificial_nodate}. Il fonctionne grâce à un grand modèle
de langage, \emph{Claude 3 Sonnet}, développé par la société
	Anthropic, et grâce au \gls{RAG}\footcite{kimaid_artificial_nodate}, technologie permettant de sélectionner des documents
	correspondant à une requête dans une base de connaissance, qui sera expliquée plus en détail dans le chapitre 5. Le futur de l'IA au service
les parlements et de leurs archives paraît prometteur. Des projets
sont en cours et une certaine d'émulation et mutualisation des
connaissances se dégagent.
\newline

À la Chambre des Députés du Grand-Duché, deux projets pilotes ont été poussés. 
Le premier est un projet de
\gls{speech}\footnote{Technologie qui convertit la parole en
	texte écrit en temps réel à l\textquotesingle aide de systèmes de
	reconnaissance vocale.}. Il est en cours, en collaboration avec
l'Université du Luxembourg. Le second, \emph{InventAIre}, est à l'origine de ce mémoire.
Ces deux projets ont
parmi leurs objectifs de prouver l'efficacité de l'IA générative sur
l'automatisation des tâches liées à l'information et au langage. Un
autre projet a été lancé afin de recenser les besoins métier spécifiques
qui pourraient être automatisés via l'IA. Ces projets permettent
également de poser les bases pour des applications futures. Actuellement,
l\textquotesingle accent est mis sur le traitement de
l\textquotesingle information, domaine où l\textquotesingle IA
générative est particulièrement efficace. Les mises en production
demeurent encore limitées et souvent davantage orientées vers la
médiation avec le public qu'au service d'applications métier
spécifiques. Cette phase d\textquotesingle expérimentation est
l'occasion d'étudier les capacités de l\textquotesingle IA et
d'identifier les prérequis nécessaires à une intégration plus large dans
les administrations parlementaires.\newline


Pour conclure ce chapitre, le contexte actuel est favorable au développement de projets
d\textquotesingle intelligence artificielle dans le secteur public. Les
ambitions des institutions, parfois élevées, se confrontent à la réalité
de l'implémentation, nécessitant des cadres régulatoires clairs. Les
initiatives récentes, comme l'\emph{AI Act} de l'Union européenne et la rédaction de
chartes internes, reflètent une volonté croissante de structurer l'usage
de l'IA. Cette dynamique
d\textquotesingle innovation et de régulation
s\textquotesingle accompagne d\textquotesingle une exploration des
applications spécifiques de l\textquotesingle IA, notamment dans la
gestion des archives. Le Luxembourg, fait face à des défis
archivistiques qui encouragent cette exploration de solutions
d\textquotesingle automatisation afin de moderniser et optimiser leur gestion et conservation.