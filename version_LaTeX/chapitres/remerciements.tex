Je tiens tout d'abord à exprimer ma gratitude envers la Cellule Archives de la Chambre des Députés pour m'avoir offert l'opportunité de réaliser ce stage. Je remercie également toutes les personnes qui m'ont accueillie chaleureusement au cours de cette expérience.
J'adresse en particulier mes remerciements à Amandine Gorse pour son encadrement attentif tout au long de mon stage ainsi que ses conseils sur la rédaction de ce mémoire. Merci à Jonathan Baud pour son aide dans la réalisation des schémas qui figurent dans ce mémoire, ainsi que pour nos discussions enrichissantes sur l'usage de l'intelligence artificielle dans les parlements.
Je remercie François-Marie Giraud pour ses recommandations bibliographiques avisées et ses réponses à mes nombreuses questions. Un grand merci également à Michel Cottin et Camille Forget des Archives nationales, ainsi qu'à Christine Mayr de la Chambre des Députés, pour leur contribution lors de la relecture de ce mémoire, notamment pour leurs commentaires pertinents et recommandations sur le deuxième chapitre.
Je remercie Florian Cafiero, mon directeur de mémoire, pour ses conseils et son accompagnement tout au long de ce travail.

Je tiens à exprimer ma reconnaissance envers ma famille pour son soutien et sa relecture attentive, et plus particulièrement à mon frère Hugo pour ses précieux éclairages concernant les systèmes d'information. Je remercie également mes amis pour leur appui. Un merci particulier à Manon pour nos après-midi de travail et nos pauses café prolongées qui m'ont été d'un grand réconfort, ainsi qu'à mes camarades de promotion, devenus des amis, pour leur soutien et nos discussions enrichissantes autour de nos stages et mémoires.


