
  	L'intelligence artificielle attire actuellement un grand afflux d'investissements, 
  	à tel point que certains craignent l'émergence d'une bulle spéculative susceptible d'éclater.
	Nous n'en sommes néanmoins pas encore là : selon son bilan publié fin août 2024, les bénéfices de l'entreprise Nvidia, qui domine le marché des
	processeurs graphiques pour l'intelligence artificielle, auraient bondi de plus de 150\% en un an.
	Les technologies d'IA promettent de stimuler la croissance en automatisant divers processus, 
	mais il reste à voir si ces investissements porteront leurs fruits à long terme, 
	en particulier dans des domaines spécifiques comme les archives.

 	
	C'est dans ce contexte que le projet \emph{InventAIre} a vu le jour à la Chambre des Députés du Grand-Duché de Luxembourg,
	 animé par la volonté d'automatiser le processus d'inventaire des fonds d'archives. 
	 Cet inventaire fournit une description de ces derniers et permet d'automatiser le calcul des délais de communicabilité.
	 Il s'agit d'un modèle provenant des Archives
	nationales du Grand-Duché (ANLux), choisi par l'équipe dans une
	volonté d'harmonisation des outils de description au niveau national.
	Cet inventaire est un fichier \emph{Excel} contenant 17 colonnes~:
	\begin{multicols}{3}
			\begin{itemize}\footnotesize{
					\item Cote
					\item Localisation
					\item Identification de la série du plan de classement
					\item Code série du tableau de tri
					\item Titre
					\item Description
					\item Période de création~: de
					\item Période de création~: à
					\item Soumis au droit d'auteur
					\item Données à caractère personnel
					\item Acte d'état civil
					\item Acte notarié
					\item Atteinte aux relations extérieures, à la sécurité du Grand-Duché ou à
					l'ordre public
					\item Affaires portées devant les instances juridictionnelles,
					extrajudiciaires ou disciplinaires
					\item Prévention, recherche de faits punissables
					\item Données commerciales et industrielles
					\item Secret fiscal}
				\end{itemize}
		\end{multicols}
	
	Le projet InventAIre est un projet pilote dont l'objectif principal
	était le développement d'un prototype permettant d'automatiser la
	rédaction de cet inventaire. Il a vu le jour à la Chambre des Députés du
	Grand-Duché, organe législatif du Luxembourg, responsable de
	l\textquotesingle élaboration et de l\textquotesingle adoption des lois.
	Composée de soixante députés élus au suffrage universel pour une durée de cinq
	ans, elle joue un rôle central dans le processus législatif et la
	surveillance du gouvernement en vertu de la séparation des pouvoirs. 
	Les fonds les plus anciens conservés par l'administation datent d'après 1945.
	L'occupant a en effet transféré l'ensemble des fonds de la Chambre aux Archives de 
	l'État en 1940. Avant ce transfert, les premiers documents conservés remontaient aux débuts de l'institution, 
	fondée en 1848, moment où une nouvelle constitution fait du Luxembourg 
	une monarchie constitutionnelle.
	\newline	
	
	Le
	projet d'automatisation du remplissage de l'inventaire de ces archives
	s'est déroulé dans le cadre de notre stage de quatre mois à la Chambre.
	Il s'agissait d'évaluer la faisabilité de cette automatisation et de
	produire un prototype d'outil.
	Nous avons également produit une note méthodologique dont le but était
	d'expliquer les choix réalisés dans le cadre du projet, les difficultés
	rencontrées, et de proposer des recommandation en cas de suite du
	projet. Elle se trouve en annexe\footnote{N’étant pas public, le document a dû être retiré de la version diffusée du mémoire. S’adresser à la Chambre des Députés pour le consulter.}. Ces réflexions ont alimenté
	la rédaction de ce mémoire, qui constitue une prise de recul
	problématisée sur le stage et les usages archivistiques potentiels de
	l'intelligence artificielle au Luxembourg. Par « usage », nous entendons
	les différentes manières dont les technologies peuvent être appliquées
	pour optimiser les processus archivistiques, qu'il s'agisse de gestion,
	conservation, communication, description ou recherche dans les archives.
	L\textquotesingle« intelligence artificielle » (IA), quant à elle, se
	réfère à des systèmes informatiques qui seraient capables de simuler des
	capacités cognitives humaines, comme l'apprentissage et la prise de
	décision, afin de traiter et analyser des données à grande échelle.
	Cependant, ce terme est parfois utilisé de manière floue pour désigner
	ce que l'on appelle plus précisément l'\gls{apprentissage}. Dans ce mémoire, nous emploierons parfois les
	termes « intelligence artificielle » et « \emph{machine learning} » de manière
	indistincte pour simplifier la compréhension, tout en étant consciente
	que le terme « intelligence artificielle » peut parfois manquer de
	précision. 
	
	Ces systèmes d'intelligence artificielle ne datent pas
	d'hier. Elle est théorisée par plusieurs penseurs tels qu'Alan Turing
	et les chercheurs Warren McCulloch et Walter Pitts, qui, dès les années
	1940, posent les bases des réseaux de neurones. Le \emph{Perceptron},
	algorithme de classification développé par Frank Rosenblatt en 1958,
	marque un premier pas dans son développement technique, introduisant un premier modèle capable
	d\textquotesingle apprendre à partir de données. Cependant, les années
	1970 et 1980 connaissent un premier «~hiver~» de l\textquotesingle IA,
	caractérisé par un manque de moyens techniques et un pessimisme
	croissant parmi les chercheurs face aux limites des technologies de
	l\textquotesingle époque. Ce déclin est suivi par une renaissance dans
	les années 1980 grâce aux systèmes experts, programmes informatiques
	conçus pour imiter le jugement et le comportement d\textquotesingle un
	expert humain dans des domaines spécifiques en utilisant des règles de
	décision et des bases de connaissances. Néanmoins, à partir de 1987, un
	second hiver de l\textquotesingle IA survient en raison des défis
	techniques persistants. Les années 1990 marquent un tournant avec des
	événements comme la défaite du champion mondial
	d\textquotesingle échecs Garry Kasparov contre \emph{Deep Blue} en 1997,
	illustrant la puissance croissante des systèmes d\textquotesingle IA.
	Une autre évolution arrive en 2008 avec l\textquotesingle émergence du
	\gls{deep}, qui transforme radicalement les capacités des
	machines, leur permettant apprendre et de traiter des données complexes. En 2016,
	\emph{AlphaGo} de Google DeepMind bat le champion du monde du jeu
	de go. Les systèmes IA commencent à être en capacité de maîtriser des jeux de
	stratégie complexes. En 2017, l\textquotesingle introduction de
	l\textquotesingle architecture \emph{Transformer} marque une nouvelle
	ère, permettant le développement de grands modèles de langage qui seront
	capables de comprendre et de générer du texte de manière plus fluide et
	contextualisée. Enfin, fin 2022, le lancement de \emph{ChatGPT} par
	\emph{OpenAI} propulse les \gls{générative}s\footnote{Branche de
			l\textquotesingle intelligence artificielle dont les modèles créent de
			nouvelles données, telles que du texte, des images ou de la musique.\newline
	
	N.B. La majorité des définitions de ce mémoire a été rédigée à l'aide de 
	ChatGPT.	
	}
	sur le devant de la scène, rendant l\textquotesingle intelligence
	artificielle plus accessible et interactive que jamais.
	Elle est de plus en plus présente dans le quotidien du grand public 
	et il en a désormais davantage conscience.
	
	Aujourd'hui, les projets se multiplient dans les secteurs publics et privés face à
	des possibilités d'automatisation qui semblent infinies. Dans le secteur
	public luxembourgeois, les projets d'intelligence artificielle
	commencent à émerger. Les administrations publiques s'intéressent à ce
	type de technologies mais peu de projets d'envergure ont pour
	l'instant abouti. Ils en sont souvent encore à la phase de pilote. 
	Ce facteur a rendu la recherche de sources difficile pour ce mémoire. 
	Nous nous sommes basée sur beaucoup de prépublications, d'articles de revues 
	ou de communications lors de conférences. Il y a encore peu d'ouvrages généraux
	 sur les enjeux et usages de l'IA dans le secteur public
	et encore moins dans le domaine archives.
	
	Les administrations ont
	néanmoins compris les avantages de l'intelligence artificielle et ont
	beaucoup d'ambition. En ce qui concerne les services d'archives publics,
	les usages de systèmes basés sur le \emph{machine learning} sont pour l'instant
	réduits. Les services doivent se concentrer sur les traitements les plus
	urgents, qui concernent le papier et sont difficilement automatisables.
	En effet, les législations sont récentes. Les services ont un arriéré
	important à traiter et manquent souvent de personnel. Au delà du défi de
	l'arriéré, les administrations doivent aussi faire face au défi du
	numérique~: la production documentaire augmente et les pratiques ne sont
	pas encore complètement formalisées. L'intelligence artificielle, via l'automatisation
	de certains processus, paraît 
	pouvoir fournir une réponse à certaines de ces problématiques.
	Cette situation soulève plusieurs
	enjeux : bien que l\textquotesingle automatisation offre de nombreuses
	possibilités et que les projets d\textquotesingle intelligence
	artificielle, très en vogue et ambitieux, soient en pleine expansion,
	les services d\textquotesingle archives publics sont-ils réellement
	prêts ? Il paraît important de déterminer les prérequis nécessaires à la
	mise en place de ce type de projets. Un travail de prise de recul sur 
	les apports, aussi divers soient-ils, et les complexités 
	de mise en place des systèmes IA s'impose.

L'intelligence artificielle est-elle une solution aux défis
archivistiques rencontrés par les producteurs d'archives publiques
luxembourgeois~?

Dans une première partie, nous verrons en quoi le contexte public
luxembourgeois est propice au lancement de projets IA malgré leur
complexité de mise en place. Les ambitions des pouvoirs publics et des
parlements sont importantes, un dialogue et un cadre se mettent en
place. Nous présenterons plus en détail les défis auxquels sont
confrontés les producteurs d'archives publiques luxembourgeois afin
d'explorer les nombreuses potentialités d'automatisation. En dépit de ce
contexte public favorable et de ces larges possibilités, la mise en
place de projets IA est complexe et nécessite des précautions.
Nous verrons ensuite qu'une fois menés, les projets IA 
peuvent avoir des apports importants pour les services d'archives. 
Ces derniers ne se situent pas forcément là où on les imagine et 
peuvent s'avérer connexes entre archives et IA.
Enfin, notre dernière partie sera consacrée aux précautions éthiques et prérequis
techniques spécifiques à l'IA. Elle sera l'occasion de réfléchir sur
des facteurs parfois ignorés à prendre en compte avant la mise en production d'outils basés sur du \emph{machine learning}.
