\chapter{Au-delà de l'automatisation, des apports connexes~: exploration des données, collaboration et visibilité}

\subsection{La quête des données d'entraînement : exploration en profondeur des données et découverte de silos à décloisonner}


	Dans la première partie du projet InventAIre, avant de choisir une
approche et de réaliser les premier tests de \emph{machine learning},
nous avons dû regrouper des données d'entraînement et de tests. Nous
avons déjà abordé le fonds bureautique du Service des relations
européennes et internationales et du protocole (SREIP) qui a été mis à
disposition. Nous avons tenté d'obtenir d'autres documents auprès du
service Technologies de l'information, qui nous a alors présenté les bases
de données de plusieurs applications. Nous avons obtenu les fichiers pdf
des documents publiés sur le site web de la Chambre\footnote{URL~: \url{https://www.chd.lu/fr/searchactivities}}.
Il s'agit de documents parlementaires publics tels que des projets de
lois, des questions parlementaires\footnote{Questions posées par les députés aux membres du gouvernement sur l'actualité politique ou bien d'intérêt général.}, les réponses à ces
questions, des procès-verbaux publics de réunions, etc. Une extraction
de la base de données du Courrier électronique de la Chambre, dont les
documents contenant des données confidentielles ont été exclus, nous a
aussi été fournie au format csv. Le Courrier électronique est un système
d'expéditions numériques interne pour le personnel et les députés. Les
expéditions contiennent des pièces-jointes stockées dans un système de fichiers et leurs métadonnées sont stockées dans une base de
données. Les pièces-jointes peuvent être des convocations, des invitations ou encore des
projets de loi. Ce sont ces documents qui ont été mis à disposition
avec certaines de leurs métadonnées. Nous avons donc eu un aperçu d'une
partie des données stockées sur les serveurs de la Chambre et de leur
organisation. Cet aperçu nous a été révélateur de l'existence d'un grand
nombre d'applications métier différentes et de silos de données.

Les silos de données sont des systèmes de gestion de
l\textquotesingle information isolés, où les données sont stockées de
manière indépendante sans communication fluide avec
d\textquotesingle autres systèmes ou applications au sein d'une même
organisation. Les inconvénients de ces silos auraient été mis en avant à
partir des années 1980 quand a commencé à se poser la question de la
collaboration entre les services\footcite{bygstad_it_2015}. Ils peuvent en effet compliquer les projets impliquant
différentes équipes qui ne travaillent pas avec les mêmes applications
et n'ont pas accès aux mêmes données. Pour les mêmes raisons, ils
peuvent néanmoins parfois constituer une forme de sécurité. Les données
sensibles de chaque équipe restent dans les mains de ces dernières ou dans celles de l'équipe qui gère les systèmes d'informations. La
question de rassembler ces données est source d'inquiétude dans un
contexte national où, comme évoqué précédemment (chapitre 2), les
préoccupations sur la divulgation de données sensibles sont importantes.
À la Chambre des Députés, l'affaire des \enquote{\emph{ChamberLeaks}} a contribué à accroître ces dernières. Il s'agit d'une faille de sécurité qui avait permis la consultation de documents internes via de simples URL en 2018. Toutefois, les archivistes devraient idéalement pouvoir archiver les
différentes données produites par les services lorsqu'elles ont une
valeur archivistique. Par exemple, à la Chambre des Députés, le contenu
du Courrier électronique peut avoir un intérêt
patrimonial, voire juridique. Des documents officiels transitent entre
le personnel de l'administration parlementaire et les députés par cette
voie. Or, son contenu est stocké dans une base de données complexe à
appréhender et les archivistes n'y ont pas accès. Les archivistes de la
Chambre n'ont pas de vue d'ensemble des différents silos et c'est également
rarement le cas dans les autres administrations. Iels ont accès aux
applications et aux données produites par leur service. En général, ce
ne sont que quelques personnes qui travaillent côté systèmes d'information
qui y ont accès et en ont une vision globale. Par ailleurs, ces silos
peuvent causer une réplication des données, ce qui augmente par la suite le
risque de désynchronisation des informations entre les différents
systèmes. Cette réplication engendre une augmentation des besoins de
stockage. La diversité des technologies et des bases de données
utilisées complique leur maintenance, rendant les processus de gestion
plus complexes et coûteux. Enfin, avant la création
d\textquotesingle une nouvelle application, il est souvent nécessaire de
recommencer l\textquotesingle étape de modélisation des données, ce qui
ralentit les projets et génère des redondances inutiles. Le
décloisonnement des silos peut donc permettre des gains financiers.
D'après Gauthier Poupeau à propos des silos qui existaient dans les
données de l'INA, «~maintenir un système et maîtriser les données est
plus difficile et plus coûteux lorsque les silos se multiplient les uns
à côté des autres, car une telle architecture dilue les
moyens\footcite{poupeau__2024}~».

A la Chambre des Députés du Grand-Duché, il existerait entre dix et
quinze silos\footnote{Le contenu de ce paragraphe est basé sur un
	entretien avec un membre du service Technologies de l'information en
	charge du projet de refonte du système d'information de la Chambre
	des Députés.}. Ils sont à la fois applicatifs et organisationnels. En
effet, chaque service a ses applications. Il en existe par exemple une
pour gérer les procès-verbaux de commission, une pour les travaux en
commission, une pour les motions, etc. Ils sont organisationnels car
chaque application répond à un besoin métier d'un service et applicatifs
parce qu'ils sont liées à une seule application. Chaque application est
basée sur des technologies propres et a sa base de données. Il y a donc
plus d'une dizaine de bases de données différentes. Il existe toutefois
une base de données métier sous la forme d'un grand référentiel qui est
connectée aux différentes applications. Elle contient par exemple un
référentiel personne, des moyens de contact, l'organigramme de la
Chambre, etc. Cette base de données connectée évite ainsi certaines
réplications de données, mais constitue aussi une application en elle-même. Face à ces silos, plusieurs solutions existent.
À la Chambre, l'utilisation de référentiels  communs à différentes applications
est une première étape afin de créer des connexions entre silos au sein
du système d'information. Un projet de \emph{data warehouse} est
également en cours. Un \emph{data warehouse} est un regroupement de
données structurées provenant de différentes sources dans le but de
faciliter leur analyse et la prise de décision au sein d'une
administration\footcite{noauthor_quest-ce_nodate},
dans un souci d'amélioration de la \emph{business
	intelligence}\footnote{Ensemble des technologies, outils et pratiques permettant de collecter, analyser et transformer des données brutes en informations pertinentes pour prendre des décisions éclairées en entreprise.}. Le projet de la Chambre des
Députés vise à répliquer les données de chaque application en
mutualisant les différentes bases de données. Ce projet a trois
objectifs principaux. La mutualisation des bases devrait faciliter le
travail de \textit{reporting}, c'est à dire de production de rapports afin de
monitorer la qualité des données. En effet, le \emph{data warehouse} devrait
permettre de réaliser différentes datavisualisations via le logiciel
d'analyses et visualisation de données \emph{Power BI} avec des modèles
de données simplifiés. Le deuxième objectif est d'octroyer aux
différents services un moyen de visualisation de leurs données,
toujours via \emph{Power BI}. Enfin, ces données réorganisées pourront
être connectées sur la plateforme d'\emph{open data} de l'État
luxembourgeois\footnote{Plateforme \url{data.public.lu}. URL des jeux de
	données publiés par la Chambre des Députés~:
	\url{https://data.public.lu/fr/organizations/chambre-des-deputes-du-grand-duche-de-luxembourg/}}
de manière automatique afin d'être publiées et téléchargeables par
l'ensemble des citoyens. Les \emph{data warehouses} et les projets de
décloisonnement des silos en général offrent ainsi des avantages
considérables pour l\textquotesingle open data et
l\textquotesingle intelligence artificielle. En centralisant et en
structurant les données, ils permettent une meilleure accessibilité et
transparence des informations pour le public. Ils permettent également
une meilleure disponibilité de jeux de données de qualité pour
l'entraînement et les tests de modèles d'IA. Ils apportent des gains de
temps sur cette première phase de collecte et préparation de données.
L'analyse facilitée par le \emph{data warehouse} garantit aussi une
meilleure qualité des données pour entraîner des modèles.

Un second projet est en marche à la Chambre de Députés. Il s'agit d'un
projet de refonte des systèmes d'information. Cette refonte a
pour but de regrouper les différentes applications métier en une seule.
L'inconvénient d'une refonte des systèmes face au problème des silos
est son coût et le temps qu'elle demande. Dans d\textquotesingle autres
administrations publiques, la refonte peut être simplifiée par un nombre
plus réduit d\textquotesingle applications et de données à restructurer.
Cette refonte est par ailleurs une occasion de mieux organiser le
\emph{records management} au sein de l'administration. À la Chambre des
Députés, le projet a donné lieu à l'identification de quatre types
principaux de données stockées dans les systèmes. Il s'agit des données
sur les acteurs, des dossiers traités, des documents de ces dossiers, et
des regroupements de personnes (réunions, évènements). Des dossiers sont effectivement traités
par des ensembles d'acteurs, qui produisent différents documents pendant
ce processus de traitement. Les documents, et idéalement l'ensemble des
données ayant une valeur archivistique devraient être conservées. Le
projet de refonte donne un regard sur ces données à archiver et fait
émerger des réflexions quand à leur archivage. Aux archives municipales
de la ville d'Amsterdam, une refonte du système informatique a
récemment été l'occasion de réfléchir à une meilleure gestion et
conservation des archives numériques. L'équipe a développé une
intégration des fonctions d'archivistiques dès la création du système
d'information\footcite{cristia_information_2020}.
Il s'agit de la méthode nommée \emph{archiving-by-design}. Cette
refonte a également permis de revoir les formats de métadonnées afin
de faciliter l'adoption du modèle de description archivistique
RiC-O\footcite{noauthor_ric_nodate}. La reprise des systèmes
d'information face au problème des silos est donc une opportunité
d'optimiser les pratiques d'archivage et de \emph{records management}.

Il existe d'autres manières de décloisonner les silos. Un deuxième
exemple est celui de l'INA (Institut national de
l\textquotesingle audiovisuel) en France. Un «~lac de données~» a été
développé dans un souci de «~recentralisation des systèmes d'information
constitués en silos\footcite{poupeau__2024}~».
Il s'agit d'un système de stockage de données brutes et hétérogènes
provenant de diverses sources au sein du système d'information. Plusieurs projets IA ont été menés à l'INA. Le décloisonnement des silos
grâce au lac de données y est également un moyen d'avoir accès à des
données propres et centralisées et à des systèmes performants pour
faciliter les automatisations via l'IA~: «~disposer d'une
infrastructure centralisée pour toutes les données de l'INA {[}lui{]}
ouvre également une nouvelle perspective : celle de pouvoir automatiser,
en s'appuyant sur des technologies d'intelligence artificielle, la
production d'un certain nombre de données\footcite{poupeau__2024}~».
La question des silos s'est aussi beaucoup posée en bibliothèque. Nous avons trouvé moins de littérature
l'abordant du point de vue des archives.
\newline

Pour conclure sur ces problématiques de silos et de
décloisonnement, la phase de collecte et de préparation des données des
projets IA donne lieu à une exploration des données présentes dans les
systèmes de l'administration qui les initie. Cette exploration peut mettre
en évidence les failles de leur organisation. Elle a ainsi des apports
pour les personnes gérant les systèmes d'information. La collaboration
avec les services gérant ces derniers dans les administrations
paraît également inévitable pour l'archivage d'une grande partie des
données qu'ils produisent. La recherche de données pour notre projet IA
a révélé une partie de la face cachée de cet iceberg des données
cloisonnées des systèmes d'information. Au delà des apports en termes
d'automatisations, les projets IA ont aussi des apports d'ordre
stratégique. Le décloisonnement des silos identifiés offre quant à lui
une opportunité de produire des données de meilleure qualité et mieux
extractibles pour entraîner ou tester des modèles de \emph{machine
	learning}. Si les archivistes sont inclus dans le processus, cela peut
avoir un impact positif sur la gestion des archives produites par les
systèmes. La découverte des silos fait aussi ressortir le silo dans
lequel se trouvent souvent les archivistes. Cantonnés à des rôles trop
précis, ils n'ont parfois pas une vision globale sur les systèmes
d'information. En effet, les archivistes ont souvent été présentés comme
des «~généralistes de l'information\footcite{banat-berger__2012}~».
D'après Françoise Banat-Berger dans un article publié en 2012 dans la
Gazette des archives intitulé «~"Un métier à part entière, l'archiviste
un généraliste de l'information" : qu'en est-il en 2012 dans le nouvel
environnement numérique ?~» , 

\begin{quote}
L'archiviste est un professionnel bien
singulier aux confins de la science de l'information, de
l'archivistique, du juridique, de la qualité, des sciences
administratives et historiques\footcite{banat-berger__2012}.
\end{quote}
\begin{quote}
	La phrase de Gérard Naud : "un métier à part entière, l'archiviste un
	généraliste de l'information" ou, comme l'écrit aujourd'hui Jean-Michel
	Salaün un "architecte de l'information", reste par conséquent tout à
	fait opérante avec une nécessité de travailler avec un périmètre de plus
	en plus large de partenaires\footcite{banat-berger__2012}.
\end{quote}

Les projets IA appliqués aux archives numériques soulignent
l\textquotesingle importance de ce rôle généraliste des archivistes,
mettant en évidence la nécessité de collaborer étroitement avec divers
acteurs, dont les informaticiens et responsables de projets numériques, pour optimiser la gestion des
documents et l'automatisation des processus archivistiques.

\subsection{Des apports moins techniques : des projets qui renforcent les liens entre les services et apportent une visibilité sur le monde des archives}

La visibilité du monde des archives et sa collaboration avec d'autres
domaines est importante pour plusieurs raisons. La question de la
collaboration est abordée dans le code de déontologie de l'\emph{ICA (International Council on Archives)}. Le dixième article stipule
que~:

\begin{quote}
	Les archivistes travaillent en collaboration avec leurs collègues et
	les membres des professions voisines afin d\textquotesingle assurer
	universellement la conservation et l\textquotesingle exploitation du
	patrimoine documentaire. Les archivistes cherchent à stimuler la
	collaboration et à éviter les conflits avec leurs collègues, en
	résolvant les difficultés par l\textquotesingle encouragement à
	respecter les normes archivistiques et l\textquotesingle éthique
	professionnelle. Les archivistes coopèrent avec les représentants des
	professions parallèles dans un esprit de respect et de compréhension
	mutuelle\footcite{noauthor_code_nodate}.
\end{quote}

La collaboration est donc une valeur fondamentale dans le domaine
archivistique. D'après Sylvie Forastier dans un article publié dans la
\emph{Gazette des archives} sur son expérience de \emph{records manager} pour un
cabinet d'avocat international au Luxembourg, «~il serait réducteur
d'envisager un service d'archives comme étant uniquement à vocation
administrative interne, un service d'archives n'est pas totalement
déconnecté du monde externe et des activités générales de
l'entreprise\footcite{forastier_archiviste_2015}~» et il « ne fonctionne pas en autarcie
mais participe à la dynamique de l'entreprise\footcite{forastier_archiviste_2015}~». Se faire connaître auprès des autres services est
effectivement essentiel pour que les données et documents à archiver de
chacun d'entre eux soient mis à disposition. Une bonne compréhension du
fonctionnement des ces services est obtenue après des discussions avec
leur personnel et permet de mieux connaître les documents produits et
leur contexte. L'archiviste est alors au fait des différents processus
de travail et peut adapter la stratégie d'archivage. Avec l'émergence
des documents numériques, l'archivistique n\textquotesingle est
aujourd'hui plus limitée aux trois âges théorisés d'Yves
Pérotin\footnote{Yves Pérotin, \enquote{L\textquotesingle administration et les \enquote{trois âges} des archives}, \emph{Seine et Paris}, 20, (octobre 1961), p. 31-33}, mais il s'agit d'une «~archivistique des flux~»\footcite{guyon_archivistique_2022}. Les archivistes doivent idéalement être impliqués dès
la création des documents numériques, «~dès l'âge courant, c'est-à-dire
dès la conception des systèmes d'information\footcite{guyon_archivistique_2022}~». Comme évoqué précédemment, les
archivistes sont désormais des «~généralistes de
l'information\footcite{banat-berger__2012}~» et ils traitent tout type
de données. Or, d'après Jenny Bunn, \emph{Head of Archives Research} aux
Archives nationales britanniques, la tendance serait à une
spécialisation du personnel, qui peut générer des silos\footcite{jaillant_are_2024}. Les administrations et les archives pourraient donc tirer
parti de collaborations interdisciplinaires plus récurrentes\footcite{jaillant_are_2024}.\newline

Par ailleurs, pour mener à bien des projets
d\textquotesingle intelligence artificielle, une collaboration étroite
avec les acteurs informatiques et les responsables de la sécurité des
systèmes d'information (RSSI) est indispensable. Les projets peuvent exploiter
l\textquotesingle expertise des chefs de projets informatiques, de
développeurs et autres professionnels de l'informatique, qui sont par la même occasion
 sensibilisés aux enjeux archivistiques, en général assez flous
pour eux. Ils peuvent alors prendre plus naturellement en compte
les préoccupations archivistiques au moment du développement informatique.
La collaboration entre archivistes et informaticiens, bien que
stimulante, est complexe, car elle nécessite une compréhension mutuelle
entre deux mondes parfois perçus comme très différents, les archives
étant parfois considérées comme une préoccupation supplémentaire non
primordiale pour les personnes issues du monde de l'informatique. La
communication peut également être délicate à mettre en place. Cette idée
est évoquée dans un article datant de 2021 intitulé «~How AI Developers
Overcome Communication Challenges in a Multidisciplinary Team: A Case
Study\footcite{piorkowski_how_2021}~». Les
équipes n'ont pas les mêmes connaissances et les personnes qui ne
maîtrisent pas l'informatique peuvent avoir de trop hautes attentes. Il
est donc important de travailler avec des personnes capables de faire le
lien entre les acteurs, qui maîtrisent à la fois le langage du monde de
l'informatique et des sciences de l'information. Des collaborations
récurrentes pourront mener à une meilleure littératie numérique chez les
archivistes et une compréhension plus approfondie des besoins de chaque
partie prenante.

Une collaboration avec d\textquotesingle autres services est également
envisageable pour obtenir les données d\textquotesingle entraînement
nécessaires. Cette collaboration renforce alors les liens en impliquant
les services qui fournissent eux-mêmes leurs données, devenant ainsi
proactifs dans le travail archivistique. Dans le cadre du projet
InventAIre, nous avons eu des liens avec les services RH et des
relations européennes, internationales et du protocole. Ils ont dès
lors été mis au courant de l'existence de notre projet. Les projets
d\textquotesingle IA apportent en effet également une certaine publicité
aux archives, même s'il convient de rester vigilant face aux risques de
mauvaise publicité en cas d'échec. Ils démontrent que le domaine
peut aussi pousser l'innovation et est utile. Nous avons par ailleurs
tenté une collaboration avec les Archives nationales du Luxembourg pour
obtenir des données de test. La collaboration peut s'étendre au-delà
d'une administration et favoriser les synergies entre administrations.
\newline 

Les contributions des projets d\textquotesingle intelligence
artificielle dans le domaine des archives vont bien au-delà de simples
automatisations. En effet, l\textquotesingle interaction avec le
personnel d\textquotesingle autres services et
l\textquotesingle exploration de leurs données peuvent être profitables
pour toutes les parties impliquées. Ces projets IA offrent par
conséquent des bénéfices variés dans des secteurs tout aussi variés.