\textbf{Résumé en français~:} 
Ce mémoire est une prise de recul après un stage de quatre mois sur un projet d'automatisation en contexte archivistique luxembourgeois.
Il examine la pertinence de l'intelligence artificielle (IA) comme moyen d'automatisation dans 
le domaine et comme solution aux défis rencontrés par les producteurs d'archives publiques au Luxembourg. Le contexte public et archivistique luxembourgeois présente des conditions favorables 
au lancement de projets d'IA.  
Les contributions potentielles de systèmes basés sur du \emph{machine learning} sont multiples 
pour les services d'archives. Elles ne se limitent pas à l'automatisation de tâches métier. 
Les apports entre le domaine des archives et de l'intelligence artificielle peuvent être connexes. 
Les ambitions sont élevées mais le déploiement d'outils basés sur ces technologies reste complexe. Des problématiques d'ordre éthique sont à prendre en compte et de nombreux prérequis techniques sont à penser en amont.

\textbf{Résumé en anglais~:} This thesis is a reflection following a four-month internship on an automation project in the Luxembourgish archival context. It examines the relevance of artificial intelligence (AI) as a means of automation in the field and as a solution to the challenges faced by public archive producers in Luxembourg. The public and archival context in Luxembourg provides favorable conditions for launching AI projects. The potential contributions of systems based on machine learning are numerous for archival services. They are not limited to the automation of business tasks. The contributions between the fields of archives and artificial intelligence can be interconnected. The ambitions are high, but the deployment of tools based on these technologies remains complex. Ethical issues must be considered, and numerous technical prerequisites need to be addressed in advance.\vspace{0.3cm}


