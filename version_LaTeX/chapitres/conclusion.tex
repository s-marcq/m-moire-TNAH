Le contexte archivistique et le contexte public luxembourgeois incitent à l'expérimentation de moyens d'automatisation. De vastes volumes d'archives papier ou numériques sont à traiter et décrire. 
Les personnes extérieures au domaine n'ont pas forcément une grande conscience de l'importance que revêtent les archives publiques, d'autant plus quand elles sont numériques.
Les projets d'intelligence artificielle dans le secteur des archives nécessitent des moyens et des connaissances techniques spécifiques. 
Pourtant, l'expérimentation de ces technologies a plusieurs avantages. 
Les projets à faible impact aboutissent plus facilement à des résultats exploitables et peuvent apporter des bénéfices en termes de médiation et de visibilité des services d'archives. 
Les initiatives d'automatisation de tâches plus complexes permettent quant à elles d'avoir une vision plus précise de certaines données à archiver et d'amorcer ou renforcer la collaboration avec les producteurs de ces données.
Le déploiement de ces technologies soulève également des questions sur l'évolution du rôle des archivistes. Ces professionnels sont pour l'instant loin d'être remplacés par des machines mais leur métier pourrait être amené à se transformer. 
Les tâches de description pourraient par exemple évoluer vers des tâches de validation d'instruments de recherche produits par IA. 
Il sera nécessaire de réfléchir à la valeur ajoutée des archivistes par rapport à celle des machines. Iels ont notamment beaucoup à apporter en matière de contextualisation et d'identification de métadonnées pertinentes\footcite{IA_INA} afin d'éviter l'effacement de certaines mémoires. Les systèmes IA, souvent décrits comme des «~boîtes noires~», soulèvent ainsi de nombreuses problématiques éthiques, notamment en termes de consommation énergétique, de pollution et à travers la question des potentiels biais dont ils peuvent être porteurs. L’archiviste a un rôle à jouer dans l’approche critique de ces technologies. Le domaine des archives dispose d’une certaine expertise face à ces enjeux, notamment en ce qui concerne la transparence et la mise à disposition de données structurées.

L’intelligence artificielle est un sujet de polarisation, oscillant entre peurs, réticences et idéalisme. Nous espérons avoir apporté dans ce mémoire une vision plus nuancée de son utilisation dans les archives. L'usage de ce type de technologies peut sembler pertinent face aux nombreux défis archivistiques au Luxembourg. Les grands modèles pré-entraînés, comme observé dans le projet InventAIre, ont un grand potentiel pour les tâches liées au langage. Cependant, les tâches archivistiques nécessitant une réflexion approfondie restent difficiles à automatiser.
Les projets à faible impact, tels que ceux visant à améliorer la médiation ou la \gls{découvrabilité} des fonds, sont davantage aisés à mettre en œuvre, surtout en l'absence de systèmes d'information archivistiques performants et de normes bien définies. Notre expérience a démontré qu’avec des moyens limités, l’automatisation d’un inventaire intellectuellement complexe n’est pas encore réalisable, bien que les \gls{LLM} montrent des capacités non négligeables en matière de résumé et d’indexation.
Notre projet a également montré que ces grands modèles de langage ne sont pas des outils « clé en main ». Leur utilisation pour des tâches complexes nécessite une réflexion approfondie. Les bénéfices de l'IA pour répondre aux défis des producteurs d'archives publiques au Luxembourg sont peut-être à chercher davantage du côté de la publicité et la médiation avec le public qu'elle peut faciliter pour les services.
Il s'agirait de capitaliser sur l'engouement actuel et l'image idéalisée de 
ces technologies.
Des bénéfices se trouvent également dans l'expérimentation que ce type de projets
motivent. L'arrivée des technologies IA est peut-être l'occasion de faire évoluer les 
pratiques archivistiques.
Au Luxembourg, elles ne sont pas encore rigoureusement normées. Cette situation peut être vue comme une opportunité. Les projets IA sont aussi un moyen d’améliorer la  \gls{litteracie} des équipes et de les sensibiliser aux spécificités des données traitées ou produites par des systèmes basés sur de l'\gls{apprentissage}. Les transformations majeures promises par ces technologies ne sont pas encore intégrées dans les processus métiers des archivistes via des automatisations, mais elles devraient commencer 
à se manifester dans les données à traiter par ces derniers.

En somme, l’intelligence artificielle n’est pas une solution miracle aux défis archivistiques luxembourgeois. Pour être efficace sur des tâches complexes, elle nécessiteraient un cadre archivistique solide, des données d’entraînement ou de test bien définies et une réflexion éthique rigoureuse. L’expérimentation dans ce domaine a toutefois des avantages certains.

Les investissements réalisés dans ces technologies peuvent avoir des apports intellectuels et les archives ont leur rôle à jouer dans 
les transformations à venir, en constituant un réservoir de connaissance et par leur capacité à contribuer à la lutte contre 
les fausses informations.
